\documentclass[9pt]{beamer}
\usetheme[progressstyle=movCircCnt]{PBEsimple}

\usepackage[T1]{fontenc}
\usepackage[utf8]{inputenc}
\usepackage[portuguese]{babel}
\usepackage{animate}
\usepackage{multicol}
\usepackage{amsfonts}
\usepackage{amssymb}
\usepackage{multirow}
\usepackage{array,booktabs}
\newcommand{\PR}[1]{\ensuremath{\left[#1\right]}}
\newcommand{\PC}[1]{\ensuremath{\left(#1\right)}}
\newcommand{\chav}[1]{\ensuremath{\left\{#1\right\}}}

\title[Bioestatística]{\bf Bioestatística\\
\vspace{.3\baselineskip}}
\subtitle[]{\bf}

\date{ 30 de Março de 2017}
\author[Isolde Previdelli]{
  Isolde Previdelli\\
  \href{itsprevidelli@uem.br}{{\tt itsprevidelli@uem.br \\
isoldeprevidelli@gmail.com \\ \vspace{8mm} \tt \textbf{\LARGE{AULA 6 -
    Distribuição discreta}}}}
}
\institute[PBE/UEM]{}

% % % % % % % % % % % % % % % % % % % % % % % % % % % % % % % % % %

\begin{document}
% Página Título
{\pbebg
\begin{frame}[plain,noframenumbering]
  \titlepage
\end{frame}}

%%%%%%%%%%%%%%
\begin{frame}{Sumário}{}
\tableofcontents
\end{frame}
%%%%%%%%%%%%%%

\section{Distribuição de Bernoulli}
%\subsection{Etapas na análise de dados}

\begin{frame}{Bernoulli}{}

Uma variável aleatória que assume apenas valores 0 e 1 com função de
probabilidade é representada por:
$$X \sim Ber(p)$$

$$
    p(x)=p^{x}(1-p)^{1-x}
$$

$$E(x)=p$$

$$Var(x)=p(1-p)$$

\end{frame}


\section{Distribuição Binomial}
\begin{frame}{Binomial}{}
Uma variável aleatória segue uma distribuição Binomial quando repetimos
um ensaio Bernoulli $n$ vezes, ou seja, quando temos uma amostra de
tamanho $n$ de uma distribuição Bernoulli.

$$X \sim Bin(n,p)$$

$$
p(x)=\binom{n}{x}  p^{x}(1-p)^{n-x}
$$


$$E(x)=np$$

$$Var(x)=np(1-p)$$


\end{frame}

\section{Distribuição Poisson}
\begin{frame}{Poisson}{}

O modelo Poisson tem sido muito utilizado em experimentos físicos e
biológicos em que $\lambda$ mede a taxa de oorrência por unidade de
medida, ou seja, o parâmetro $\lambda$ é frequência média ou esperada de
ocorrências num determinado intervalo de tempo.

$$X \sim Pois(\lambda)$$

$$p(x)=\displaystyle\frac{e^{-\lambda}\lambda^{x}}{x!}$$

$$E(x)=\lambda$$

$$Var(x)=\lambda$$

\end{frame}


%\section{Distribuição Poison}
\begin{frame}{Exemplos}{}

\textbf{Alguns exemplos para aplicação das distribuições Bernoulli,
Binomial e Poisson podem ser vistos no endereço abaixo:\\}
\indent

\url{https://biostatistics-uem.github.io/Bio/probabilidade.html}

\end{frame}

%%%%%%%%%%%%%%%

{\pbebg
\begin{frame}[plain,noframenumbering]

\finalpage{\begin{figure}[!htb]
    \centering
   \includegraphics[scale=0.475]{probII.jpg}
  \end{figure}
Obrigada!}

\end{frame}}
%%%%%%%%%%%%%%%%
\end{document}